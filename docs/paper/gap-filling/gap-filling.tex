\documentclass[12pt]{article}

\usepackage{libs/template-latex/sbc-template}
\usepackage{graphicx,url}
\usepackage[utf8]{inputenc}
\usepackage[brazil]{babel}
\usepackage{caption}
\usepackage{subcaption}
\usepackage{float}
\usepackage{amsmath}
\usepackage{booktabs}

\graphicspath{{assets/}}

\sloppy

\title{Preenchimento de Lacunas (Gap-Filling) em DTMs HiRISE usando Inferência Monocular com Vision Transformers (ViT)}

\author{Bruno R. B. F. Holanda\inst{1} }

\address{Instituto do Hardware BR – (HBR)\\
	Av. Alan Turing, n°: 776 – 13.083-898 – Campinas – SP – Brasil
	\email{brunorodriguesholanda@gmail.com}
}

\begin{document}
	
	\maketitle
	
	\begin{abstract}
		numero de palavras 150 - 250
	\end{abstract}
	
	\begin{resumo}
		Numero de palavras: 150 - 250 - igual ao abstract porém em pt-BR
		Espaço Visual (Páginas): 0.5
		Foco Principal: Venda o peixe: Problema + Solução (ViT) + Melhor Resultado.
	\end{resumo}
	
	\section{Introdução}
	
	A exploração da superfície de Marte atingiu um novo patamar de detalhamento com a câmera \textit{High Resolution Imaging Science Experiment} (HiRISE), a bordo da sonda \textit{Mars Reconnaissance Orbiter} (MRO). Os produtos gerados por este instrumento, especificamente os Modelos Digitais de Terreno (DTMs), oferecem uma resolução espacial na ordem de metros por pixel, constituindo um recurso inestimável para a comunidade científica. Estes dados são fundamentais para estudos geomorfológicos de precisão, modelagem de processos de superfície e, crucialmente, para o planejamento estratégico e segurança de missões robóticas, como o rover Perseverance \cite{mcewen2007mars}.
	
	A metodologia predominante para a geração destes modelos topográficos é a fotogrametria estéreo, que reconstrói a estrutura 3D da superfície através da identificação de pontos homólogos em pares de imagens orbitais \cite{kirk2011overview}. Apesar de sua robustez geométrica, esta técnica possui limitações intrínsecas severas. O processo de correspondência de pixels (\textit{stereo matching}) falha sistematicamente em condições adversas, notadamente em regiões de textura homogênea — como vastos campos de dunas — ou em áreas de iluminação extrema, caracterizadas por sombras profundas ou reflexos especulares em gelo. O resultado direto dessas falhas é a presença massiva de lacunas de dados (pixels com valor \textit{NoData}) nos produtos finais.
	
	A descontinuidade causada por estas lacunas compromete a integridade dos dados, impedindo análises espaciais contínuas essenciais, como simulações de fluxo hidrológico e cálculos de declividade. Tradicionalmente, a mitigação deste problema recorre a métodos de interpolação matemática, como a bilinear ou \textit{splines}. No entanto, tais abordagens tendem a gerar superfícies artificialmente suaves, ignorando a morfologia local e resultando em artefatos geologicamente implausíveis. A Figura \ref{fig:comparacao_inicial} ilustra este cenário, contrastando a riqueza de textura visual disponível na ortoimagem com a ausência de informação altimétrica no DTM correspondente.
	
	\begin{figure}[H]
		\centering
		\begin{subfigure}[b]{0.45\textwidth}
			\centering
			\includegraphics[height=5cm, keepaspectratio]{intro_imagem_orthoretificada.png}
			\caption{Ortoimagem HiRISE (Textura)}
			\label{fig:ortho}
		\end{subfigure}
		\hfill
		\begin{subfigure}[b]{0.45\textwidth}
			\centering
			\includegraphics[height=5cm, keepaspectratio]{intro_dtm_HiRISE.png}
			\caption{DTM com Lacunas (Preto = NoData)}
			\label{fig:dtm}
		\end{subfigure}
		\caption{Exemplo do problema abordado: (a) A imagem visual contém informações completas de textura e sombreamento, enquanto (b) o DTM derivado via estéreo apresenta falhas significativas (áreas pretas) onde a correlação falhou.}
		\label{fig:comparacao_inicial}
	\end{figure}
	
	Para superar essas limitações, este trabalho propõe uma abordagem baseada em Aprendizado Profundo (\textit{Deep Learning}), especificamente no campo da Estimativa de Profundidade Monocular (MDE). A premissa central é que a ortoimagem monocular (Figura \ref{fig:ortho}), que permanece íntegra mesmo onde a fotogrametria falha, contém pistas visuais suficientes — através de sombreamento (\textit{Shape-from-Shading}) e textura — para inferir a topografia subjacente.
	
	Embora trabalhos seminais como o MADNet \cite{tao2021madnet} tenham demonstrado a viabilidade do uso de Redes Adversariais Generativas (GANs) baseadas em convoluções (U-Net) para esta tarefa, propõe-se o uso de uma arquitetura mais recente e promissora: os \textit{Vision Transformers} (ViT). Modelos como o DPT (\textit{Dense Prediction Transformer}) \cite{ranftl2021vision} utilizam mecanismos de auto-atenção global, permitindo capturar relações de longo alcance na imagem que são frequentemente perdidas por janelas de convolução limitadas. A hipótese deste estudo é que um modelo ViT-DPT pode aprender a complexa função de transferência entre o albedo marciano e sua topografia de forma mais coesa que as abordagens anteriores.
	
	Portanto, os objetivos deste trabalho são: (1) consolidar um conjunto de dados curado de pares Ortoimagem-DTM contendo apenas dados válidos para treinamento supervisionado; (2) treinar uma arquitetura ViT-DPT para prever a topografia a partir da informação visual; e (3) aplicar o modelo treinado para preencher seletivamente as lacunas \textit{NoData} em produtos oficiais, gerando DTMs híbridos "prontos para análise". Esta pesquisa visa entregar uma solução automatizada que aumente significativamente a usabilidade científica do vasto arquivo de dados da missão MRO.
	
	\section{Metodologia}
	
	A metodologia estruturou-se em três etapas sequenciais: a curadoria de um conjunto de dados livre de falhas para treinamento supervisionado, a adaptação e treinamento da arquitetura \textit{Vision Transformer} (ViT), e o desenvolvimento de um pipeline de inferência com pós-processamento para a fusão topográfica.
	
	\subsection{Aquisição e Preparação de Dados}
	
	Para viabilizar o aprendizado supervisionado, consolidou-se um conjunto de dados a partir do \textit{Planetary Data System} (PDS). Pares de Modelos Digitais de Terreno (DTMs) e Ortoimagens das missões PSP e ESP foram indexados e processados.
	
	Inicialmente, cada DTM foi reprojetado e alinhado pixel a pixel com sua respectiva ortoimagem utilizando a biblioteca GDAL, aplicando reamostragem cúbica para minimizar artefatos. Em seguida, implementou-se uma estratégia de recorte (\textit{tiling}) com janela deslizante de $512 \times 512$ pixels e passo (\textit{stride}) de 256 pixels (Figura \ref{fig:tiling_process}).
	
	Um filtro de qualidade rigoroso foi aplicado: apenas recortes contendo 100\% de pixels válidos de elevação foram mantidos, descartando-se qualquer amostra com valores \textit{NoData}. Isso garantiu que o modelo aprendesse exclusivamente com a "verdade terrestre" fotogramétrica. Os dados resultantes foram normalizados para o intervalo $[0, 1]$ e serializados em formato \textit{Parquet} para otimização de I/O, divididos em conjuntos de treino (80\%) e validação (20\%).
	
	\begin{figure}[H]
		\centering
		\includegraphics[width=0.7\textwidth, keepaspectratio]{metodologia_dtm_title_stride.png}
		\caption{Estratégia de extração e recorte.}
		\label{fig:tiling_process}
	\end{figure}
	
	\begin{figure}[H]
		\centering
		\includegraphics[width=1.0\textwidth, height=0.85\textheight, keepaspectratio]{metodologia_dataset_sequence.png}
		\caption{Fluxo detalhado de aquisição, limpeza e preparação dos dados (Dataset Pipeline).}
		\label{fig:seq_dataset}
	\end{figure}
	
	\subsection{Arquitetura de Rede e Treinamento}
	
	Adotou-se a arquitetura \textit{Dense Prediction Transformer} (DPT) (Figura \ref{fig:dpt_arch}), utilizando o modelo \texttt{Intel/dpt-large} pré-treinado no dataset MiDaS. Esta abordagem substitui redes convolucionais puras por mecanismos de atenção global, permitindo correlacionar texturas locais com o contexto geomorfológico amplo.
	
	\begin{figure}[H]
		\centering
		\includegraphics[width=0.99\textwidth, keepaspectratio]{metodologia_dpt_architecture.jpg}
		\caption{Arquitetura DPT. Adaptado de \cite{ranftl2021vision}.}
		\label{fig:dpt_arch}
	\end{figure}
	
	O treinamento foi orquestrado via \textit{PyTorch} com \textit{Distributed Data Parallel} (DDP) em 4 GPUs A10G de 24 GB cada (Seção \ref{subsec:hardware}). Utilizou-se o otimizador \textbf{AdamW} (decaimento de peso de 0.01) e precisão mista (AMP). Para evitar \textit{overfitting}, aplicou-se \textit{Early Stopping} com paciência de 5 épocas monitorando a perda de validação.
	
	Para capturar a complexidade topográfica, definiu-se uma Função de Perda Combinada ($L_{total}$), conforme a Equação \ref{eq:loss}:
	
	\begin{equation} \label{eq:loss}
		L_{total} = 0.6 \cdot \mathcal{L}_{L1} + 0.3 \cdot \mathcal{L}_{grad} + 0.1 \cdot \mathcal{L}_{SSIM}
	\end{equation}
	
	Esta função pondera o erro absoluto ($\mathcal{L}_{L1}$), a consistência de bordas e inclinações através de operadores de Sobel ($\mathcal{L}_{grad}$), e a similaridade estrutural perceptual ($\mathcal{L}_{SSIM}$).
	
	\begin{figure}[H]
		\centering
		\includegraphics[width=1.0\textwidth, height=0.85\textheight, keepaspectratio]{metodologia_train_sequence.png}
		\caption{Sequência de operações durante uma época de treinamento supervisionado.}
		\label{fig:seq_train}
	\end{figure}

	\subsection{Infraestrutura de Hardware} \label{subsec:hardware}
	
	Todos os experimentos foram executados em infraestrutura local seguindo as recomendações do projeto: (i) a geração de dados (ETL) foi feita com 4 núcleos de CPU, 32 GB de RAM e 2 TB de armazenamento; (ii) o treinamento utilizou \textit{Distributed Data Parallel} em 4 GPUs NVIDIA A10G (24 GB cada, total de 96 GB), 48 núcleos de CPU, 192 GB de RAM e 2 TB de disco; (iii) a inferência de preenchimento foi validada em uma GPU NVIDIA T4 de 24 GB, com 4 núcleos de CPU, 16 GB de RAM e 100 GB de armazenamento. A sincronização e os tempos reportados na Seção de Resultados referem-se a essa configuração.
	
	\subsection{Pipeline de Inferência e Fusão}
	
	O preenchimento das lacunas não se resumiu à inferência direta. Desenvolveu-se um pipeline robusto para garantir consistência altimétrica e visual.
	
	\subsubsection{Inferência Contextual e Denormalização}
	Para mitigar efeitos de borda, adotou-se uma inferência com contexto expandido. Embora o bloco de predição final fosse de $512 \times 512$ pixels, a entrada da rede recebia uma área de $768 \times 768$ pixels (com margem de 128px).
	
	Como o modelo estima profundidade relativa normalizada ($Z_{pred}$), foi necessário convertê-la para a altitude absoluta marciana ($Z_{final}$). Aplicou-se um alinhamento estatístico linear (Equação \ref{eq:denorm}) baseado na média ($\mu$) e desvio padrão ($\sigma$) dos pixels válidos na interseção entre a predição e o DTM original:
	
	\begin{equation} \label{eq:denorm}
		Z_{final} = Z_{pred} \cdot \left( \frac{\sigma_{real}}{\sigma_{pred} + \epsilon} \right) + \left( \mu_{real} - \mu_{pred} \cdot \frac{\sigma_{real}}{\sigma_{pred} + \epsilon} \right)
	\end{equation}
	
	\subsubsection{Fusão de Bordas (Blending)}
	Para eliminar descontinuidades na junção entre o pixel real e o sintético, aplicou-se um algoritmo de fusão. Uma máscara das lacunas foi dilatada e suavizada por filtro Gaussiano ($\sigma=2.0$), criando uma zona de transição gradual (\textit{feathering}) que interpolou os valores, garantindo continuidade topográfica.
	
	\begin{figure}[H]
		\centering
		\includegraphics[width=1.0\textwidth, height=0.85\textheight, keepaspectratio]{metodologia_fill_sequence.png}
		\caption{Pipeline de inferência: da imagem bruta à fusão topográfica.}
		\label{fig:seq_fill}
	\end{figure}
	
	\subsection{Protocolo de Validação}
	
	Devido à inexistência de dados reais nas áreas de lacuna, a validação utilizou o método de "Lacunas Sintéticas". Máscaras de falha artificiais foram introduzidas em DTMs de teste originalmente perfeitos. As métricas RMSE (Erro Quadrático Médio), MAE (Erro Absoluto Médio) e SSIM foram calculadas comparando-se a reconstrução da IA com o dado original oculto.
	
	\section{Resultados}
	
	O modelo foi treinado por 7 épocas em modo distribuído (DDP) sobre 4 GPUs A10G de 24 GB cada (Seção \ref{subsec:hardware}). A avaliação considerou tanto a dinâmica de aprendizado quanto a precisão final em dados de teste nunca vistos pelo modelo.
	
	\subsection{Dinâmica de Treinamento e Convergência}
	
	A análise das curvas de aprendizado demonstra uma convergência robusta. A Figura \ref{fig:convergencia_loss} ilustra o decaimento da função de perda total ao longo das iterações (batches). Observa-se uma redução exponencial inicial, seguida de uma estabilização assintótica, indicando que o modelo aprendeu efetivamente a mapear as características visuais em valores de profundidade.
	
	\begin{figure}[H]
		\centering
		\includegraphics[width=0.9\textwidth]{resultado_convergencia_loss.png}
		\caption{Evolução da Perda Total (Loss) por batch. A linha suave indica a tendência média, mostrando aprendizado contínuo sem instabilidades severas.}
		\label{fig:convergencia_loss}
	\end{figure}
	
	Para verificar a capacidade de generalização e ausência de \textit{overfitting}, monitorou-se o erro nos conjuntos de treino e validação (Figura \ref{fig:treino_validacao}). A perda de validação acompanhou a tendência de queda da perda de treino, estabilizando-se em torno de 0.12, o que valida a capacidade do modelo de inferir topografia em imagens novas.
	
	\begin{figure}[H]
		\centering
		\includegraphics[width=0.8\textwidth]{resultado_treino_vs_validacao.png}
		\caption{Curvas de perda média por época: Treino vs. Validação. A proximidade das curvas indica boa generalização.}
		\label{fig:treino_validacao}
	\end{figure}
	
	A Figura \ref{fig:componentes} detalha o comportamento das componentes da função de custo. A estabilização das perdas de Gradiente e SSIM confirma que a rede não apenas minimizou o erro métrico, mas também aprendeu a reproduzir a alta frequência espacial (textura e bordas) do terreno marciano.
	
	\begin{figure}[H]
		\centering
		\includegraphics[width=0.8\textwidth]{resultado_componentes_medios_por_epoca.png}
		\caption{Contribuição média das componentes de perda (L1, Gradiente, SSIM) ao longo das épocas.}
		\label{fig:componentes}
	\end{figure}
	
	\subsection{Avaliação em Dados de Teste}
	
	O modelo treinado foi aplicado a três classes geomorfológicas distintas. A Tabela \ref{tab:metrics} apresenta as métricas quantitativas obtidas.
	
	\begin{table}[H]
		\centering
		\caption{Métricas de desempenho em dados de teste (Ground Truth).}
		\label{tab:metrics}
		\begin{tabular}{lcccc}
			\toprule
			\textbf{Terreno} & \textbf{RMSE (m)} & \textbf{SSIM} & \textbf{Tempo (s)} \\ 
			\midrule
			Dunas (Dunes)    & 14.67 & 0.9996 & 1.79 \\
			Planícies (Plains) & 3.33  & 0.9998 & 2.52 \\
			Escarpas (Scarps)  & 15.21 & 0.9998 & 1.79 \\ 
			\midrule
			\textbf{Média} & \textbf{11.07} & \textbf{0.9997} & \textbf{2.03} \\ 
			\bottomrule
		\end{tabular}
	\end{table}
	
	Visualmente, a reconstrução apresenta alta fidelidade. A Figura \ref{fig:qualitativo} compara a entrada (com lacuna simulada), a predição da rede e a verdade terrestre. O mapa de erro evidencia que as maiores discrepâncias ocorrem nas cristas das dunas, onde a variação de declividade é abrupta.
	
	\begin{figure}[H]
		\centering
		\begin{subfigure}[b]{0.48\textwidth}
			\includegraphics[width=\textwidth]{resultado_dtm_dunas_com_nodata.jpg}
			\caption{Entrada (Lacuna)}
		\end{subfigure}
		\hfill
		\begin{subfigure}[b]{0.48\textwidth}
			\includegraphics[width=\textwidth]{resultado_dtm_dunas_preenchido.jpg}
			\caption{Predição (ViT)}
		\end{subfigure}
		
		\vspace{0.2cm}
		
		\begin{subfigure}[b]{0.48\textwidth}
			\includegraphics[width=\textwidth]{resultado_dtm_dunas.jpg}
			\caption{Ground Truth}
		\end{subfigure}
		\hfill
		\begin{subfigure}[b]{0.48\textwidth}
			\includegraphics[width=\textwidth]{resultado_dtm_dunas_error_map.jpg}
			\caption{Mapa de Erro}
		\end{subfigure}
		\caption{Resultados qualitativos em campo de dunas. O modelo recupera a morfologia das ondulações (b) de forma consistente com o original (c).}
		\label{fig:qualitativo}
	\end{figure}
	
	\section{Análise e Discussão}
	
	Os resultados indicam que a abordagem baseada em ViT é superior aos métodos de interpolação clássica na preservação da morfologia. O alto valor de SSIM ($>0.99$) em todas as classes confirma que a estrutura visual do terreno é recuperada com precisão quase perfeita.
	
	Contudo, observa-se uma dependência da complexidade do terreno na precisão altimétrica absoluta. Em **Planícies**, onde a variação topográfica é suave, o RMSE é extremamente baixo (3.33 m). Em **Dunas** e **Escarpas**, o erro sobe para ~15 m. Isso se deve à ambiguidade de escala na inferência monocular: o modelo deduz corretamente a \textit{forma} (frequência das dunas), mas pode ter um viés (\textit{offset}) na amplitude vertical absoluta em áreas de sombra complexa.
	
	A análise do perfil topográfico (Figura \ref{fig:perfil}) corrobora esta análise: as curvas de predição e real são paralelas e morfologicamente idênticas, apresentando apenas deslocamentos verticais locais. Para aplicações científicas, isso significa que a análise de declividade e rugosidade — cruciais para a navegação de rovers — é preservada, mesmo que a altitude absoluta tenha margem de erro.
	
	\begin{figure}[H]
		\centering
		\includegraphics[width=0.8\textwidth]{profile_graph.jpg}
		\caption{Perfil topográfico comparativo. A linha vermelha (Predição) segue fielmente a tendência e frequência da linha azul (Real).}
		\label{fig:perfil}
	\end{figure}
	
	\section{Conclusão}
	
	Este trabalho apresentou um pipeline inédito utilizando \textit{Vision Transformers} para o preenchimento de lacunas em DTMs de Marte. Os resultados confirmam que a arquitetura DPT é capaz de aprender a complexa função de transferência entre albedo e topografia. O método mostrou-se computacionalmente eficiente (~2s por tile) e morfologicamente preciso. Trabalhos futuros focarão na inclusão de altimetria laser (MOLA) como restrição extra para corrigir o viés de escala em terrenos acidentados.
	
	\bibliographystyle{libs/template-latex/sbc}
	\bibliography{references}
	
\end{document}

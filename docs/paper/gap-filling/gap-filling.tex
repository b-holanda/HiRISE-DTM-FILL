\documentclass[12pt]{article}

\usepackage{libs/template-latex/sbc-template}
\usepackage{graphicx,url}
\usepackage[utf8]{inputenc}
\usepackage[brazil]{babel}
\usepackage{caption}
\usepackage{subcaption}
\usepackage{float}
\usepackage{amsmath}
\usepackage{booktabs}
\usepackage{microtype}
\usepackage{setspace} % Para melhor controle de espaçamento se necessário

\graphicspath{{assets/}}

\sloppy

\title{Preenchimento de Lacunas (Gap-Filling) em DTMs HiRISE usando Inferência Monocular com Vision Transformers (ViT)}

\author{Bruno R. B. F. Holanda\inst{1}}

\address{Instituto do Hardware BR – (HBR)\\
	Av. Alan Turing, n°: 776 – 13.083-898 – Campinas – SP – Brasil
	\email{brunorodriguesholanda@gmail.com}
}

\begin{document}
	
	\maketitle
	
	\begin{abstract}
		The HiRISE stereo pipeline occasionally fails over homogeneous textures and cast shadows, leaving critical gaps in Digital Terrain Models (DTMs) that degrade traverse planning and hazard analysis for Mars rovers. We propose a monocular gap-filling approach based on the Dense Prediction Transformer (ViT-DPT), leveraging global self-attention to recover topography from single orthoimages where convolutional methods underperform. A curated dataset of $512 \times 512$ HiRISE ortho/DTM tiles without \textit{NoData} was used to fine-tune \texttt{Intel/dpt-large} with a composite loss under distributed training. The inference pipeline adds contextual padding, statistical denormalization, and Gaussian blending to ensure altimetric consistency. On held-out tiles, the method achieved mean SSIM of 0.9997 and per-tile latency of approximately 2 s, yielding analysis-ready DTMs. Although RMSE reaches 14--15 m in dunes due to monocular scale ambiguity, morphology and local slope cues are preserved, enabling reliable mobility assessment. The approach delivers mission-compatible terrain products at low computational cost.
	\end{abstract}
	
	\begin{resumo}
		Lacunas em DTMs HiRISE geradas por falhas de correspondência estéreo em texturas homogêneas e sombras comprometem a navegação e a segurança de rovers em Marte. Este trabalho apresenta um preenchimento monocular baseado no \textit{Dense Prediction Transformer} (ViT-DPT), cuja atenção global supera limitações de CNNs ao explorar pistas de sombreamento e textura em ortoimagens únicas. Um conjunto curado de tiles $512 \times 512$ sem \textit{NoData} foi usado para ajustar o modelo \texttt{Intel/dpt-large} com perda combinada em treinamento distribuído. O pipeline de inferência agrega \textit{padding} contextual, desnormalização estatística e fusão Gaussiana para garantir consistência altimétrica. Em dados de teste, o método obteve SSIM médio de 0,9997 e latência de cerca de 2 s por tile, produzindo DTMs prontos para análise de trafegabilidade. Embora o RMSE alcance 14--15 m em dunas devido à ambiguidade de escala monocular, a morfologia e as inclinações locais são preservadas, o que é essencial para engenharia de mobilidade. A solução entrega produtos compatíveis com operações de missão a baixo custo computacional.
	\end{resumo}
	
	\section{Introdução}
	
	A exploração da superfície de Marte atingiu um novo patamar de detalhamento com a câmera \textit{High Resolution Imaging Science Experiment} (HiRISE), a bordo da sonda \textit{Mars Reconnaissance Orbiter} (MRO). Os produtos gerados por este instrumento, especificamente os Modelos Digitais de Terreno (DTMs), oferecem uma resolução espacial na ordem de 1 metro por pixel. Este nível de detalhe constitui um recurso inestimável para a comunidade científica, permitindo estudos geomorfológicos de alta precisão e modelagem de processos eólicos e hidrológicos passados. Crucialmente, estes dados são o alicerce para o planejamento estratégico e a segurança operacional de missões robóticas de superfície, como os rovers \textit{Perseverance} e \textit{Curiosity} \cite{mcewen2007mars}.
	
	A metodologia predominante para a geração destes modelos topográficos é a fotogrametria estéreo, que reconstrói a estrutura tridimensional da superfície através da identificação de pontos homólogos em pares de imagens orbitais adquiridas em ângulos distintos \cite{kirk2011overview}. Apesar de sua robustez geométrica comprovada ao longo de décadas, esta técnica possui limitações intrínsecas severas quando aplicada a superfícies planetárias não cooperativas.
	
	O processo de correspondência de pixels (\textit{stereo matching}) falha sistematicamente em condições adversas, notadamente em: (1) regiões de textura homogênea, como vastos campos de dunas ou mantos de areia lisa, onde não há contraste suficiente para correlacionar pixels; e (2) áreas de iluminação extrema, caracterizadas por sombras profundas em crateras ou reflexos especulares em formações de gelo. O resultado direto dessas falhas é a presença massiva de lacunas de dados (pixels com valor \textit{NoData}) nos produtos finais.
	
	A descontinuidade causada por estas lacunas compromete gravemente a integridade dos dados para engenharia de sistemas. Algoritmos de navegação autônoma e planejamento de trajetórias dependem de mapas de custo contínuos para calcular inclinação (\textit{slope}) e rugosidade. A presença de "buracos" cegos no mapa força os planejadores de missão a assumirem o pior caso ou a traçarem rotas sub-ótimas para contornar áreas desconhecidas. Tradicionalmente, a mitigação deste problema recorre a métodos de interpolação matemática, como a bilinear ou \textit{splines}. No entanto, tais abordagens tendem a gerar superfícies artificialmente suaves, ignorando a morfologia local (como as ondulações de uma duna) e resultando em artefatos geologicamente e fisicamente implausíveis.
	
	A Figura \ref{fig:comparacao_inicial} ilustra dramaticamente este cenário, contrastando a riqueza de textura visual disponível na ortoimagem com a ausência total de informação altimétrica no DTM correspondente.
	
	\begin{figure}[htbp]
		\centering
		\begin{subfigure}[b]{0.48\textwidth}
			\centering
			\includegraphics[width=\textwidth, keepaspectratio]{intro_imagem_orthoretificada.png}
			\caption{Ortoimagem HiRISE (Textura)}
			\label{fig:ortho}
		\end{subfigure}
		\hfill
		\begin{subfigure}[b]{0.48\textwidth}
			\centering
			\includegraphics[width=\textwidth, keepaspectratio]{intro_dtm_HiRISE.png}
			\caption{DTM com Lacunas (NoData)}
			\label{fig:dtm}
		\end{subfigure}
		\caption{O problema da correlação estéreo: (a) A imagem visual contém informações completas de textura e sombreamento, enquanto (b) o DTM derivado apresenta falhas significativas (áreas pretas) onde o algoritmo de correlação não convergiu.}
		\label{fig:comparacao_inicial}
	\end{figure}
	
	Para superar essas limitações, este trabalho propõe uma abordagem baseada em Aprendizado Profundo (\textit{Deep Learning}), especificamente no campo da Estimativa de Profundidade Monocular (MDE). A premissa central é que a ortoimagem monocular (Figura \ref{fig:ortho}), que permanece íntegra mesmo onde a fotogrametria falha, contém pistas visuais suficientes — através de sombreamento (\textit{Shape-from-Shading}) e padrões de textura — para inferir a topografia subjacente.
	
	Embora trabalhos seminais como o MADNet \cite{tao2021madnet} tenham demonstrado a viabilidade do uso de Redes Convolucionais (CNNs) para esta tarefa, propõe-se aqui o uso de uma arquitetura de última geração: os \textit{Vision Transformers} (ViT). Modelos como o DPT (\textit{Dense Prediction Transformer}) \cite{ranftl2021vision} utilizam mecanismos de auto-atenção global, permitindo capturar relações de longo alcance na imagem que são frequentemente perdidas por janelas de convolução limitadas.
	
	Os objetivos deste trabalho são: (1) consolidar um conjunto de dados curado de pares Ortoimagem-DTM contendo apenas dados válidos ("Ground Truth"); (2) treinar uma arquitetura ViT-DPT para prever a topografia a partir da informação visual monocular; e (3) aplicar o modelo treinado em um pipeline de produção para preencher seletivamente as lacunas em produtos oficiais, gerando DTMs híbridos "prontos para análise".
	
	\section{Metodologia}
	
	A metodologia estruturou-se em três etapas sequenciais: a curadoria de um conjunto de dados robusto, a adaptação da arquitetura Transformer, e o desenvolvimento de um pipeline de inferência tolerante a falhas.
	
	\subsection{Aquisição e Preparação de Dados}
	
	Para viabilizar o aprendizado supervisionado, consolidou-se um conjunto de dados a partir do \textit{Planetary Data System} (PDS). Pares de Modelos Digitais de Terreno (DTMs) e Ortoimagens das missões PSP e ESP foram indexados e processados. Cada DTM foi reprojetado e alinhado pixel a pixel com sua respectiva ortoimagem utilizando ferramentas geoespaciais (GDAL), garantindo co-registro preciso.
	
	Implementou-se uma estratégia de recorte (\textit{tiling}) com janela deslizante de $512 \times 512$ pixels e passo (\textit{stride}) de 256 pixels, gerando uma sobreposição de 50\% (Figura \ref{fig:tiling_process}). Esta sobreposição atua como uma técnica de \textit{data augmentation} natural, permitindo que o modelo veja as mesmas feições topográficas em diferentes posições relativas dentro do quadro.
	
	\begin{figure}[htbp]
		\centering
		\includegraphics[width=0.85\textwidth, keepaspectratio]{metodologia_dtm_title_stride.png}
		\caption{Estratégia de extração e recorte. A sobreposição (stride) garante maior volume de dados e robustez nas bordas.}
		\label{fig:tiling_process}
	\end{figure}
	
	Um filtro de qualidade rigoroso foi aplicado: apenas recortes contendo 100\% de pixels válidos de elevação foram mantidos, descartando-se qualquer amostra com valores \textit{NoData}. Isso garantiu que o modelo aprendesse exclusivamente com a "verdade terrestre" fotogramétrica confiável. Os dados resultantes foram normalizados para o intervalo $[0, 1]$ e serializados em formato \textit{Parquet} para otimização de I/O durante o treinamento distribuído.
	
	\subsection{Arquitetura de Rede: Por que Transformers?}
	
	Adotou-se a arquitetura \textit{Dense Prediction Transformer} (DPT) (Figura \ref{fig:dpt_arch}), inicializada com pesos \texttt{Intel/dpt-large}. A escolha por Transformers em detrimento de CNNs clássicas (como U-Net) é fundamentada na capacidade de modelagem de contexto global.
	
	Em uma CNN, o campo receptivo efetivo cresce lentamente com a profundidade da rede. Isso significa que, para inferir a altura de uma duna, a rede "olha" apenas para a vizinhança imediata. Se a duna for muito larga e homogênea, a CNN perde a referência de onde a duna começa e termina. O \textit{Vision Transformer}, através do mecanismo de auto-atenção (\textit{Self-Attention}), permite que cada pixel (ou *patch* da imagem) atenda a todos os outros pixels simultaneamente desde as primeiras camadas. Isso é crucial para terrenos marcianos, onde a inclinação local depende do contexto geológico macroscópico.
	
	\begin{figure}[htbp]
		\centering
		\includegraphics[width=1.0\textwidth, keepaspectratio]{metodologia_dpt_architecture.jpg}
		\caption{Arquitetura DPT. A substituição do encoder convolucional por um Transformer permite que o modelo capture contexto global, essencial para resolver ambiguidades em texturas homogêneas. Adaptado de \cite{ranftl2021vision}.}
		\label{fig:dpt_arch}
	\end{figure}
	
	Para o treinamento, definiu-se uma Função de Perda Combinada ($L_{total}$) projetada para preservar tanto a altimetria quanto a morfologia:
	\begin{equation} \label{eq:loss}
		L_{total} = 0.6 \cdot \mathcal{L}_{L1} + 0.3 \cdot \mathcal{L}_{grad} + 0.1 \cdot \mathcal{L}_{SSIM}
	\end{equation}
	
	Onde $\mathcal{L}_{L1}$ penaliza erros de altura absoluta, $\mathcal{L}_{grad}$ penaliza diferenças na inclinação (derivada primeira) — vital para rovers — e $\mathcal{L}_{SSIM}$ garante a fidelidade estrutural da imagem gerada.
	
	\subsection{Infraestrutura Computacional} \label{subsec:hardware}
	
	O treinamento de modelos de visão modernos exige capacidade computacional significativa. Foi utilizada uma estratégia de treinamento distribuído (\textit{Distributed Data Parallel} - DDP) para acelerar a convergência e permitir o uso de *batches* maiores. A Tabela \ref{tab:hardware} detalha a especificação do hardware empregado nos experimentos.
	
	\begin{table}[htbp]
		\centering
		\caption{Especificação da Infraestrutura de Hardware utilizada.}
		\label{tab:hardware}
		\begin{tabular}{lll}
			\toprule
			\textbf{Componente} & \textbf{Especificação} & \textbf{Função no Pipeline} \\ 
			\midrule
			GPU Cluster & 4$\times$ NVIDIA A10G (24GB VRAM) & Treinamento Distribuído \\
			GPU Inference & 1$\times$ NVIDIA T4 (16GB VRAM) & Validação e Inferência \\
			CPU & AMD EPYC (48 vCPUs) & Pré-processamento e DataLoader \\
			RAM & 192 GB DDR4 & Cache de Datasets em Memória \\
			Armazenamento & 2 TB NVMe SSD & I/O de Alta Performance \\
			\bottomrule
		\end{tabular}
	\end{table}
	
	\subsection{Pipeline de Inferência e Fusão}
	
	O preenchimento das lacunas utiliza um fluxo de pós-processamento robusto. Para mitigar efeitos de borda nas junções dos *tiles*, adotou-se uma inferência com contexto expandido: a entrada da rede é de $768 \times 768$ pixels, mas apenas o centro de $512 \times 512$ é utilizado, descartando as bordas instáveis.
	
	Como a saída do ViT é uma profundidade relativa (adimensional e normalizada), é necessário reintroduzir a escala física. Isso é feito através de um alinhamento estatístico linear (Equação \ref{eq:denorm}) baseado na média ($\mu$) e desvio padrão ($\sigma$) dos pixels válidos na borda da lacuna:
	
	\begin{equation} \label{eq:denorm}
		Z_{final} = Z_{pred} \cdot \left( \frac{\sigma_{real}}{\sigma_{pred} + \epsilon} \right) + \left( \mu_{real} - \mu_{pred} \cdot \frac{\sigma_{real}}{\sigma_{pred} + \epsilon} \right)
	\end{equation}
	
	Finalmente, uma fusão Gaussiana (\textit{blending}) é aplicada na interface entre os dados reais e os preenchidos, garantindo uma transição suave e derivável, requisito obrigatório para simuladores de dinâmica veicular.
	
	\section{Resultados}
	
	O modelo foi treinado por 7 épocas completas. A análise das curvas de aprendizado demonstra convergência estável e rápida.
	
	\begin{figure}[htbp]
		\centering
		\includegraphics[width=0.85\textwidth]{resultado_treino_vs_validacao.png}
		\caption{Curvas de perda média por época: Treino vs. Validação. A proximidade e o decaimento paralelo das curvas indicam ausência de \textit{overfitting} e boa capacidade de generalização para dados não vistos.}
		\label{fig:treino_validacao}
	\end{figure}
	
	\subsection{Custos e Viabilidade Econômica}
	
	A análise de custos é fundamental para a viabilidade de processamento em escala planetária (todo o arquivo de Marte). O custo total do projeto na AWS (período out/nov 2025) foi de US\$ 1.940,49. A maior parte deste valor (aprox. 68\%) foi alocada em instâncias EC2 Spot para o treinamento. O custo de inferência mostrou-se marginal, permitindo estimar que o processamento de um DTM HiRISE completo custaria apenas alguns centavos de dólar. Isso valida o método como uma solução economicamente escalável para agências espaciais.
	
	\subsection{Avaliação em Dados de Teste}
	
	Para quantificar o desempenho, utilizou-se o protocolo de "Lacunas Sintéticas", onde áreas conhecidas foram apagadas e reconstruídas. A Tabela \ref{tab:metrics} apresenta as métricas por tipo de terreno.
	
	\begin{table}[htbp]
		\centering
		\caption{Métricas de desempenho quantitativo em dados de teste (Ground Truth).}
		\label{tab:metrics}
		\begin{tabular}{lccc}
			\toprule
			\textbf{Classe de Terreno} & \textbf{RMSE (m)} & \textbf{SSIM} & \textbf{Latência (s)} \\ 
			\midrule
			Dunas (Dunes)    & 14.67 & 0.9996 & 1.79 \\
			Planícies (Plains) & 3.33  & 0.9998 & 2.52 \\
			Escarpas (Scarps)  & 15.21 & 0.9998 & 1.79 \\ 
			\midrule
			\textbf{Média Global} & \textbf{11.07} & \textbf{0.9997} & \textbf{2.03} \\ 
			\bottomrule
		\end{tabular}
	\end{table}
	
	Visualmente, a reconstrução apresenta alta fidelidade (Figura \ref{fig:qualitativo}). O mapa de erro evidencia que as maiores discrepâncias ocorrem nas cristas das dunas, onde a variação de declividade é abrupta, mas a estrutura geral é preservada.
	
	\begin{figure}[htbp]
		\centering
		\begin{subfigure}[b]{0.48\textwidth}
			\includegraphics[width=\textwidth]{resultado_dtm_dunas_com_nodata.jpg}
			\caption{Entrada (Lacuna Simulada)}
		\end{subfigure}
		\hfill
		\begin{subfigure}[b]{0.48\textwidth}
			\includegraphics[width=\textwidth]{resultado_dtm_dunas_preenchido.jpg}
			\caption{Predição (ViT)}
		\end{subfigure}
		
		\vspace{0.2cm}
		
		\begin{subfigure}[b]{0.48\textwidth}
			\includegraphics[width=\textwidth]{resultado_dtm_dunas.jpg}
			\caption{Ground Truth (Real)}
		\end{subfigure}
		\hfill
		\begin{subfigure}[b]{0.48\textwidth}
			\includegraphics[width=\textwidth]{resultado_dtm_dunas_error_map.jpg}
			\caption{Mapa de Erro Absoluto}
		\end{subfigure}
		\caption{Resultados qualitativos em campo de dunas complexo. O modelo recupera a morfologia das ondulações (b) de forma visualmente consistente com o original (c).}
		\label{fig:qualitativo}
	\end{figure}
	
	\section{Análise e Discussão}
	
	Os resultados obtidos permitem uma discussão aprofundada sobre as compensações entre precisão métrica absoluta e fidelidade morfológica, especialmente no contexto de exploração aeroespacial.
	
	\subsection{Morfologia vs. Altimetria Absoluta}
	
	O dado mais impactante é a discrepância entre o RMSE (~14m em dunas) e o SSIM (>0.999). O RMSE alto em terrenos acidentados decorre da ambiguidade de escala inerente à visão monocular (\textit{scale ill-posedness}). Sem uma referência estéreo ou laser, o modelo infere a "forma" correta das dunas baseada no sombreamento, mas pode errar a amplitude vertical exata ou o nível base (offset).
	
	No entanto, para a segurança de rovers, a \textbf{morfologia} é frequentemente mais crítica que a altitude absoluta. Algoritmos de planejamento de trajetória (como A* ou D*) calculam o custo de travessia baseados na inclinação (\textit{slope}) e na rugosidade do terreno. A Figura \ref{fig:perfil} demonstra que, embora exista um deslocamento vertical, as derivadas do terreno (a forma das curvas) são preservadas.
	
	\begin{figure}[htbp]
		\centering
		\includegraphics[width=0.95\textwidth]{profile_graph.jpg}
		\caption{Perfil topográfico comparativo. As discrepâncias de altura são sistemáticas (offset), mas a rugosidade e a tendência de inclinação são preservadas, validando o uso para análise de trafegabilidade.}
		\label{fig:perfil}
	\end{figure}
	
	Isso significa que um rover planejado sobre o DTM inferido "sentiria" as mesmas inclinações e obstáculos que no terreno real, permitindo a detecção de perigos de deslizamento ou tombamento, mesmo que a coordenada Z absoluta esteja deslocada.
	
	\section{Conclusão}
	
	Este trabalho apresentou um pipeline completo e validado para o preenchimento de lacunas em DTMs de Marte utilizando \textit{Vision Transformers}. Diferente de métodos de interpolação simples que mascaram perigos ao suavizar o terreno, a abordagem baseada em IA recupera a textura rugosa e a morfologia das feições geológicas.
	
	Os experimentos demonstraram que, apesar do desafio de escala em visão monocular, a fidelidade estrutural (SSIM 0.9997) é suficiente para gerar produtos "prontos para análise". O sistema é robusto, computacionalmente eficiente para processamento em nuvem e economicamente viável.
	
	Para trabalhos futuros, recomenda-se a integração de dados esparsos de altimetria a laser (MOLA) como "âncoras" durante a inferência. Isso permitiria corrigir o viés vertical observado nas dunas, unindo a precisão absoluta do laser com a resolução espacial e contextual do Transformer, aproximando-se do "Estado da Arte" definitivo para cartografia planetária automatizada.
	
	\bibliographystyle{libs/template-latex/sbc}
	\bibliography{references}
	
\end{document}

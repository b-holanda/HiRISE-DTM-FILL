\documentclass[12pt]{article}

\usepackage{libs/template-latex/sbc-template}
\usepackage{graphicx,url}
\usepackage[utf8]{inputenc}
\usepackage[brazil]{babel}
\usepackage{caption}
\usepackage{subcaption}
\usepackage{float}
\usepackage{amsmath} % Adicionado para fórmulas matemáticas

\graphicspath{{assets/}}

\sloppy

\title{Preenchimento de Lacunas (Gap-Filling) em DTMs HiRISE usando Inferência Monocular com Vision Transformers (ViT)}

\author{Bruno R. B. F. Holanda\inst{1} }

\address{Instituto do Hardware BR – (HBR)\\
	Av. Alan Turing, n°: 776 – 13.083-898 – Campinas – SP – Brasil
	\email{brunorodriguesholanda@gmail.com}
}

\begin{document} 
	
	\maketitle
	
	\begin{abstract}
		numero de palavras 150 - 250
	\end{abstract}
	
	\begin{resumo} 
		Numero de palavras: 150 - 250 - igual ao abstract porém em pt-BR
		Espaço Visual (Páginas): 0.5
		Foco Principal: Venda o peixe: Problema + Solução (ViT) + Melhor Resultado.
	\end{resumo}
	
	\section{Introdução}
	
	A exploração da superfície de Marte atingiu um novo patamar de detalhamento com a câmera \textit{High Resolution Imaging Science Experiment} (HiRISE), a bordo da sonda \textit{Mars Reconnaissance Orbiter} (MRO). Os produtos gerados por este instrumento, especificamente os Modelos Digitais de Terreno (DTMs), oferecem uma resolução espacial na ordem de metros por pixel, constituindo um recurso inestimável para a comunidade científica. Estes dados são fundamentais para estudos geomorfológicos de precisão, modelagem de processos de superfície e, crucialmente, para o planejamento estratégico e segurança de missões robóticas, como o rover Perseverance (MCEWEN et al., 2007).
	
	A metodologia predominante para a geração destes modelos topográficos é a fotogrametria estéreo, que reconstrói a estrutura 3D da superfície através da identificação de pontos homólogos em pares de imagens orbitais (KIRK et al., 2011). Apesar de sua robustez geométrica, esta técnica possui limitações intrínsecas severas. O processo de correspondência de pixels (\textit{stereo matching}) falha sistematicamente em condições adversas, notadamente em regiões de textura homogênea — como vastos campos de dunas — ou em áreas de iluminação extrema, caracterizadas por sombras profundas ou reflexos especulares em gelo. O resultado direto dessas falhas é a presença massiva de lacunas de dados (pixels com valor \textit{NoData}) nos produtos finais.
	
	A descontinuidade causada por estas lacunas compromete a integridade dos dados, impedindo análises espaciais contínuas essenciais, como simulações de fluxo hidrológico e cálculos de declividade. Tradicionalmente, a mitigação deste problema recorre a métodos de interpolação matemática, como a bilinear ou \textit{splines}. No entanto, tais abordagens tendem a gerar superfícies artificialmente suaves, ignorando a morfologia local e resultando em artefatos geologicamente implausíveis. A Figura \ref{fig:comparacao_inicial} ilustra este cenário, contrastando a riqueza de textura visual disponível na ortoimagem com a ausência de informação altimétrica no DTM correspondente.
	
	\begin{figure}[H] 
		\centering
		\begin{subfigure}[b]{0.35\textwidth}
			\centering
			\includegraphics[width=\textwidth]{intro_imagem_orthoretificada.png}
			\caption{Ortoimagem HiRISE (Textura)}
			\label{fig:ortho}
		\end{subfigure}
		\hspace{1cm}
		\begin{subfigure}[b]{0.35\textwidth}
			\centering
			\includegraphics[width=\textwidth]{intro_dtm_HiRISE.png}
			\caption{DTM com Lacunas (Preto = NoData)}
			\label{fig:dtm}
		\end{subfigure}
		\caption{Exemplo do problema abordado: (a) A imagem visual contém informações completas de textura e sombreamento, enquanto (b) o DTM derivado via estéreo apresenta falhas significativas (áreas pretas) onde a correlação falhou.}
		\label{fig:comparacao_inicial}
	\end{figure}
	
	Para superar essas limitações, este trabalho propõe uma abordagem baseada em Aprendizado Profundo (\textit{Deep Learning}), especificamente no campo da Estimativa de Profundidade Monocular (MDE). A premissa central é que a ortoimagem monocular (Figura \ref{fig:ortho}), que permanece íntegra mesmo onde a fotogrametria falha, contém pistas visuais suficientes — através de sombreamento (\textit{Shape-from-Shading}) e textura — para inferir a topografia subjacente.
	
	Embora trabalhos seminais como o MADNet (TAO et al., 2021) tenham demonstrado a viabilidade do uso de Redes Adversariais Generativas (GANs) baseadas em convoluções (U-Net) para esta tarefa, propõe-se o uso de uma arquitetura mais recente e promissora: os \textit{Vision Transformers} (ViT). Modelos como o DPT (\textit{Dense Prediction Transformer}) (RANFTL et al., 2021) utilizam mecanismos de auto-atenção global, permitindo capturar relações de longo alcance na imagem que são frequentemente perdidas por janelas de convolução limitadas. A hipótese deste estudo é que um modelo ViT-DPT pode aprender a complexa função de transferência entre o albedo marciano e sua topografia de forma mais coesa que as abordagens anteriores.
	
	Portanto, os objetivos deste trabalho são: (1) consolidar um conjunto de dados curado de pares Ortoimagem-DTM contendo apenas dados válidos para treinamento supervisionado; (2) treinar uma arquitetura ViT-DPT para prever a topografia a partir da informação visual; e (3) aplicar o modelo treinado para preencher seletivamente as lacunas \textit{NoData} em produtos oficiais, gerando DTMs híbridos "prontos para análise". Esta pesquisa visa entregar uma solução automatizada que aumente significativamente a usabilidade científica do vasto arquivo de dados da missão MRO.
	
	\section{Metodologia}
	
	A metodologia estruturou-se em três etapas sequenciais: a curadoria de um conjunto de dados livre de falhas para treinamento supervisionado, a adaptação e treinamento da arquitetura \textit{Vision Transformer} (ViT), e o desenvolvimento de um pipeline de inferência com pós-processamento para a fusão topográfica.
	
	\subsection{Aquisição e Preparação de Dados}
	
	Para viabilizar o aprendizado supervisionado, consolidou-se um conjunto de dados a partir do \textit{Planetary Data System} (PDS). Pares de Modelos Digitais de Terreno (DTMs) e Ortoimagens das missões PSP e ESP foram indexados e processados.
	
	Inicialmente, cada DTM foi reprojetado e alinhado pixel a pixel com sua respectiva ortoimagem utilizando a biblioteca GDAL, aplicando reamostragem cúbica para minimizar artefatos. Em seguida, implementou-se uma estratégia de recorte (\textit{tiling}) com janela deslizante de $512 \times 512$ pixels e passo (\textit{stride}) de 256 pixels (Figura \ref{fig:tiling_process}).
	
	Um filtro de qualidade rigoroso foi aplicado: apenas recortes contendo 100\% de pixels válidos de elevação foram mantidos, descartando-se qualquer amostra com valores \textit{NoData}. Isso garantiu que o modelo aprendesse exclusivamente com a "verdade terrestre" fotogramétrica. Os dados resultantes foram normalizados para o intervalo $[0, 1]$ e serializados em formato \textit{Parquet} para otimização de I/O, divididos em conjuntos de treino (80\%) e validação (20\%).
	
	\begin{figure}[H]
		\centering
		\includegraphics[width=0.6\textwidth]{metodologia_dtm_title_stride.png}
		\caption{Estratégia de extração.}
		\label{fig:tiling_process}
	\end{figure}
	
	\subsection{Arquitetura de Rede e Treinamento}
	
	Adotou-se a arquitetura \textit{Dense Prediction Transformer} (DPT) (Figura \ref{fig:dpt_arch}), utilizando o modelo \texttt{Intel/dpt-large} pré-treinado no dataset MiDaS. Esta abordagem substitui redes convolucionais puras por mecanismos de atenção global, permitindo correlacionar texturas locais com o contexto geomorfológico amplo.
	
	\begin{figure}[H]
		\centering
		\includegraphics[width=0.6\textwidth]{metodologia_dpt_architecture.jpg}
		\caption{Arquitetura DPT Adaptado de Ranftl et al. (2021).}
		\label{fig:dpt_arch}
	\end{figure}
	
	O treinamento foi orquestrado via \textit{PyTorch} com \textit{Distributed Data Parallel} (DDP). Utilizou-se o otimizador \textbf{AdamW} (decaimento de peso de 0.01) e precisão mista (AMP). Para evitar \textit{overfitting}, aplicou-se \textit{Early Stopping} com paciência de 5 épocas monitorando a perda de validação.
	
	Para capturar a complexidade topográfica, definiu-se uma Função de Perda Combinada ($L_{total}$), conforme a Equação \ref{eq:loss}:
	
	\begin{equation} \label{eq:loss}
		L_{total} = 0.6 \cdot \mathcal{L}_{L1} + 0.3 \cdot \mathcal{L}_{grad} + 0.1 \cdot \mathcal{L}_{SSIM}
	\end{equation}
	
	Esta função pondera o erro absoluto ($\mathcal{L}_{L1}$), a consistência de bordas e inclinações através de operadores de Sobel ($\mathcal{L}_{grad}$), e a similaridade estrutural perceptual ($\mathcal{L}_{SSIM}$).
	
	\subsection{Pipeline de Inferência e Fusão}
	
	O preenchimento das lacunas não se resumiu à inferência direta. Desenvolveu-se um pipeline robusto para garantir consistência altimétrica e visual.
	
	\subsubsection{Inferência Contextual e Denormalização}
	Para mitigar efeitos de borda, adotou-se uma inferência com contexto expandido. Embora o bloco de predição final fosse de $512 \times 512$ pixels, a entrada da rede recebia uma área de $768 \times 768$ pixels (com margem de 128px).
	
	Como o modelo estima profundidade relativa normalizada ($Z_{pred}$), foi necessário convertê-la para a altitude absoluta marciana ($Z_{final}$). Aplicou-se um alinhamento estatístico linear (Equação \ref{eq:denorm}) baseado na média ($\mu$) e desvio padrão ($\sigma$) dos pixels válidos na interseção entre a predição e o DTM original:
	
	\begin{equation} \label{eq:denorm}
		Z_{final} = Z_{pred} \cdot \left( \frac{\sigma_{real}}{\sigma_{pred} + \epsilon} \right) + \left( \mu_{real} - \mu_{pred} \cdot \frac{\sigma_{real}}{\sigma_{pred} + \epsilon} \right)
	\end{equation}
	
	\subsubsection{Fusão de Bordas (Blending)}
	Para eliminar descontinuidades na junção entre o pixel real e o sintético, aplicou-se um algoritmo de fusão. Uma máscara das lacunas foi dilatada e suavizada por filtro Gaussiano ($\sigma=2.0$), criando uma zona de transição gradual (\textit{feathering}) que interpolou os valores, garantindo continuidade topográfica.
	
	\subsection{Protocolo de Validação}
	
	Devido à inexistência de dados reais nas áreas de lacuna, a validação utilizou o método de "Lacunas Sintéticas". Máscaras de falha artificiais foram introduzidas em DTMs de teste originalmente perfeitos. As métricas RMSE (Erro Quadrático Médio), MAE (Erro Absoluto Médio) e SSIM foram calculadas comparando-se a reconstrução da IA com o dado original oculto.
	
	\section{Resultados}
	
	Numero de palavras: 400 - 600 
	Espaço Visual (Páginas): 1.5 a 2.0
	Foco Principal: Crucial: Menos texto, mais tabelas de métricas e figuras comparativas.
	
	\section{Análise e Discução}
	
	Numero de palavras: 700 - 900
	Espaço Visual (Páginas): 1.5 a 2.0
	Foco Principal: Crucial: O "porquê" dos resultados. Comparação com estado da arte.
	
	\section{Conclusão}
	
	Numero de palavras: 200 - 300
	Espaço Visual (Páginas): 0.5
	Foco Principal: Crucial: Retomada dos objetivos, limitações e trabalhos futuros.
	
\end{document}
